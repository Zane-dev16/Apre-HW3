\documentclass{article}

\usepackage{amsmath, amsthm, amssymb, amsfonts}
\usepackage{thmtools}
\usepackage{graphicx}
\usepackage{setspace}
\usepackage{geometry}
\usepackage{float}
\usepackage{hyperref}
\usepackage[utf8]{inputenc}
\usepackage[english]{babel}
\usepackage{framed}
\usepackage[dvipsnames]{xcolor}
\usepackage[most]{tcolorbox}
\usepackage{minted}
\usepackage{enumitem}

\usepackage{indentfirst}

\usepackage[export]{adjustbox} % Align images

\colorlet{LightGray}{White!90!Periwinkle}
\colorlet{LightOrange}{Orange!15}
\colorlet{LightGreen}{Green!15}

\newcommand{\HRule}[1]{\rule{\linewidth}{#1}}

\newtcbtheorem[auto counter,number within=section]{code}{Código}{
  colback=LightOrange!20,
  colframe=LightOrange,
  colbacktitle=LightOrange,
  fonttitle=\bfseries\color{black},
  boxed title style={size=small,colframe=LightOrange},
}{code}

\setstretch{1.2}
\geometry{
  textheight=22.5cm,
  textwidth=13.75cm,
  top=2.5cm,
  headheight=12pt,
  headsep=25pt,
  footskip=30pt
}

% ------------------------------------------------------------------------------

\begin{document}

% ------------------------------------------------------------------------------
% Cover Page and ToC
% ------------------------------------------------------------------------------
\begin{center}
  \begin{figure}
    \includegraphics[scale = 0.3, left]{img/IST_A.eps} % IST logo
    \end{figure}
  \LARGE{ \normalsize \textsc{} \\
  [2.0cm] 
  \LARGE{ \LARGE \textsc{Aprendizagem}} \\
  [1cm]
  \LARGE{ \LARGE \textsc{LEIC IST-UL}} \\
  [1cm]
  \HRule{1.5pt} \\
  [0.4cm]
  \LARGE \textbf{\uppercase{Relatório - Homework 1}}
  \HRule{1.5pt}
  \\ [2.5cm]
  }
\end{center}

\begin{flushleft}
  \textbf{\LARGE Grupo 10:}
\end{flushleft}

\begin{center}
  \begin{minipage}{0.7\textwidth}
      \begin{flushleft}
        \large Gabriel Ferreira \\
        \large  Irell Zane
      \end{flushleft}
  \end{minipage}%
  \begin{minipage}{0.3\textwidth}
      \begin{flushright}
        \large 107030\\
        \large 107161
      \end{flushright}
  \end{minipage}
\end{center}

\begin{center}
  \vspace{4cm}
  \date \large \bf  2024/2025 -- 1st Semester, P1
\end{center}

\setcounter{page}{0}
\thispagestyle{empty}
\renewcommand{\thesection}{\Roman{section}}

\newpage

% ------------------------------------------------------------------------------
% Content
% ------------------------------------------------------------------------------



\large{\textbf{Part I}: Pen and paper}\normalsize

\begin{enumerate}[leftmargin=\labelsep]
\item Question summary can go here.
    \begin{enumerate}
    \item Place your solution. Math can be entered using the equation
    environment like this
    \begin{equation}
        \vec{\mathbf{r}} = \vec{\mathbf{r}}_{0} + \vec{\mathbf{v}}_{0}t + \frac{1}{2}\vec{\mathbf{a}}t^{2}
    \end{equation}
    If you then where working in say the $x$-direction and had some numbers % A percent sign allows you to comment.
    %The dollar signs around something in a line of text is for "in-line math"
    \begin{equation}
    \begin{array}{r@{~=~}l}
    x & x_{0} + v_{x0}t + \frac{1}{2}a_{x}t^{2} \\ [2ex]
    & 1.2~\text{m} + (4.0~\text{m/s})(3.0~\text{s}) + \frac{1}{2}(-1.0~\text{m/s}^{2})(3.0~\text{s})^{2}\\ [2ex]
    & \boxed{8.7~\text{m}}
    \end{array}
    \end{equation}

    \item When you get to the next part, you can add a \verb"\item" to get the appropriate label. Also,
    if you don't like all the equation numbers, you can use the following to have the equation with
    no number
    \begin{equation*}
    \sum\vec{\mathbf{F}} = m\vec{\mathbf{a}}
    \end{equation*}

    \item For more details on putting math into {\LaTeX} documents you can see 
    \href{https://www.overleaf.com/learn/latex/Mathematical_expressions}{this page on Overleaf}.
    \end{enumerate}

\item We you get to the next problem, you can end the enumerate for the parts of the previous problem and then add another item.
    \begin{enumerate}
    \item Use a nested enumerate environment to label the parts of the next problem.
    \item For a quick and broad overview of how to create documents in {\LaTeX} see 
    \href{https://www.overleaf.com/learn/latex/Learn_LaTeX_in_30_minutes}{this quick tutorial from Overleaf}.
    \end{enumerate}
\end{enumerate}

\large{\textbf{Part II}: Programming}\normalsize

\begin{enumerate}[leftmargin=\labelsep,resume]
\item Solution to the programming questions here.
\end{enumerate}

\vskip 1cm
\textbf{End note}: do not forget to also submit your Jupyter notebook

\newpage

% ----------------------------------------------------------------------
% Cover
% ----------------------------------------------------------------------

\end{document}
