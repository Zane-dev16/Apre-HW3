\documentclass{article}

\usepackage{amsmath, amsthm, amssymb, amsfonts}
\usepackage{thmtools}
\usepackage{graphicx}
\usepackage{setspace}
\usepackage{geometry}
\usepackage{float}
\usepackage{hyperref}
\usepackage[utf8]{inputenc}
\usepackage[english]{babel}
\usepackage{framed}
\usepackage[dvipsnames]{xcolor}
\usepackage[most]{tcolorbox}
\usepackage{minted}
\usepackage{enumitem}

\usepackage{indentfirst}

\usepackage[export]{adjustbox} % Align images

\colorlet{LightGray}{White!90!Periwinkle}
\colorlet{LightOrange}{Orange!15}
\colorlet{LightGreen}{Green!15}

\newcommand{\HRule}[1]{\rule{\linewidth}{#1}}

\newtcbtheorem[auto counter,number within=section]{code}{Código}{
  colback=LightOrange!20,
  colframe=LightOrange,
  colbacktitle=LightOrange,
  fonttitle=\bfseries\color{black},
  boxed title style={size=small,colframe=LightOrange},
}{code}

\setstretch{1.2}
\geometry{
  textheight=22.5cm,
  textwidth=13.75cm,
  top=2.5cm,
  headheight=12pt,
  headsep=25pt,
  footskip=30pt
}

% ------------------------------------------------------------------------------

\begin{document}

% ------------------------------------------------------------------------------
% Cover Page and ToC
% ------------------------------------------------------------------------------
\begin{center}
  \begin{figure}
    \includegraphics[scale = 0.3, left]{img/IST_A.eps} % IST logo
    \end{figure}
  \LARGE{ \normalsize \textsc{} \\
  [2.0cm] 
  \LARGE{ \LARGE \textsc{Aprendizagem}} \\
  [1cm]
  \LARGE{ \LARGE \textsc{LEIC IST-UL}} \\
  [1cm]
  \HRule{1.5pt} \\
  [0.4cm]
  \LARGE \textbf{\uppercase{Relatório - Homework 1}}
  \HRule{1.5pt}
  \\ [2.5cm]
  }
\end{center}

\begin{flushleft}
  \textbf{\LARGE Grupo 10:}
\end{flushleft}

\begin{center}
  \begin{minipage}{0.7\textwidth}
      \begin{flushleft}
        \large Gabriel Ferreira \\
        \large  Irell Zane
      \end{flushleft}
  \end{minipage}%
  \begin{minipage}{0.3\textwidth}
      \begin{flushright}
        \large 107030\\
        \large 107161
      \end{flushright}
  \end{minipage}
\end{center}

\begin{center}
  \vspace{4cm}
  \date \large \bf  2024/2025 -- 1st Semester, P1
\end{center}

\setcounter{page}{0}
\thispagestyle{empty}
\renewcommand{\thesection}{\Roman{section}}

\newpage

% ------------------------------------------------------------------------------
% Content
% ------------------------------------------------------------------------------



\large{\textbf{Part I}: Pen and paper}\normalsize

\begin{enumerate}[leftmargin=\labelsep]
\item 

Given the polynomial basis function:

\[\phi(y_1, y_2) = y_1 \times y_2\]

We apply this to our input data:

\begin{align*}
x_1: \phi(1, 1) &= 1 \times 1 = 1 \\
x_2: \phi(1, 3) &= 1 \times 3 = 3 \\
x_3: \phi(3, 2) &= 3 \times 2 = 6 \\
x_4: \phi(3, 3) &= 3 \times 3 = 9 \\
x_5: \phi(2, 4) &= 2 \times 4 = 8
\end{align*}

For OLS, we need $\mathbf{X}$ (input) and $\mathbf{y}$ (output) matrices:

\[\mathbf{X} = \begin{bmatrix}
1 & 1 \\
1 & 3 \\
1 & 6 \\
1 & 9 \\
1 & 8
\end{bmatrix}\]

\[\mathbf{z} = \begin{bmatrix}
1.25 \\
7.0 \\
2.7 \\
3.2 \\
5.5
\end{bmatrix}\]

\textbf{OLS closed form solution calculation}

The OLS closed form solution is given by:

\[\boldsymbol{w} = (\mathbf{X}^T \mathbf{X})^{-1} \mathbf{X}^T \mathbf{z}\]

Calculation of the separate components of formula:

\[
\mathbf{X}^T \mathbf{X} = 
\begin{bmatrix}
1 & 1 & 1 & 1 & 1 \\
1 & 3 & 6 & 9 & 8
\end{bmatrix}
\begin{bmatrix}
1 & 1 \\
1 & 3 \\
1 & 6 \\
1 & 9 \\
1 & 8
\end{bmatrix}
=
\begin{bmatrix}
5 & 27 \\
27 & 191
\end{bmatrix}
\]

\[(\mathbf{X}^T \mathbf{X})^{-1} \approx \begin{bmatrix}
0.845 & -0.119 \\
-0.119 & 0.022
\end{bmatrix}\]

\[\mathbf{X}^T \mathbf{z} = \begin{bmatrix}
19.65 \\
111.25
\end{bmatrix}\]

Finally,

\[\boldsymbol{w} = (\mathbf{X}^T \mathbf{X})^{-1} \mathbf{X}^T \mathbf{z}\]
\[= \begin{bmatrix}
0.845 & -0.119 \\
-0.119 & 0.022
\end{bmatrix} \times \begin{bmatrix}
19.65 \\
111.25
\end{bmatrix}\]

\[\approx \begin{bmatrix}
3.316\\
0.114
\end{bmatrix}\]

Therefore, the regression model in the transformed space is:

\[y_{num} = 3.316 + 0.114 \times \phi(y_1, y_2)\]

\item We you get to the next problem, you can end the enumerate for the parts of the previous problem and then add another item.
    \begin{enumerate}
    \item Use a nested enumerate environment to label the parts of the next problem.
    \item For a quick and broad overview of how to create documents in {\LaTeX} see 
    \href{https://www.overleaf.com/learn/latex/Learn_LaTeX_in_30_minutes}{this quick tutorial from Overleaf}.
    \end{enumerate}
\end{enumerate}

\large{\textbf{Part II}: Programming}\normalsize

\begin{enumerate}[leftmargin=\labelsep,resume]
\item Solution to the programming questions here.
\end{enumerate}

\vskip 1cm
\textbf{End note}: do not forget to also submit your Jupyter notebook

\newpage

% ----------------------------------------------------------------------
% Cover
% ----------------------------------------------------------------------

\end{document}

